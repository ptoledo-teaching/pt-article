%%
%% PT Article Template
%% Full-featured template demonstrating all capabilities
%%
%% Author: Pedro Toledo Correa
%% Version: 0.1
%% Date: 2025-10-18
%%

\documentclass{pt-article}
% Available options:
% - Language: spanish (default), english, portuguese, french
% - Code: nominted (disable minted, use verbatim)
% - Standard article options: twocolumn, onecolumn, 10pt, 11pt, 12pt

% Document metadata
\title{Título del Artículo}
\titlesub{Subtítulo Opcional}
\titlesubsub{Sub-subtítulo Opcional}

% Version control
\version{1.0}
\build{auto}  % Use 'auto' for automatic build counting or specify a number
% \watermark{BORRADOR}  % Uncomment to add watermark

% Authors (add as many as needed)
\addauthor{Nombre}{Apellido}{correo@ejemplo.com}{Departamento, Universidad}
\addauthor{Segundo}{Autor}{segundo.autor@universidad.cl}{Instituto de Investigación}

% Optional: Add institutional logo
% \logo{}

% Academic/Class information (optional)
\classcode{IWI-131}
\classname{Programación}
\classsemester{Primer Semestre 2025}

% Institution information (optional)
\department{Departamento de Informática}
\school{Escuela de Ingeniería}
\university{Universidad Técnica Federico Santa María}

\begin{document}

% Title is automatically generated at the beginning

% Abstract (optional)
\begin{abstract}
Este es el resumen del artículo. Debe proporcionar una descripción concisa
del contenido, objetivos, metodología y principales conclusiones del trabajo.
Típicamente no debe exceder las 250 palabras.
\end{abstract}

% Table of contents (optional)
% \tableofcontents
% \ptlistoffigures  % Only shown if figures exist
% \ptlistoftables   % Only shown if tables exist
% \ptlistofcodes    % Only shown if code listings exist

\section{Introducción}

Esta es una plantilla completa que demuestra todas las capacidades de la clase
\inlinecode{pt-article}. El documento está configurado en formato de dos columnas
con márgenes optimizados y espaciado profesional.

\subsection{Características Principales}

La clase \inlinecode{pt-article} extiende la clase estándar \inlinecode{article}
de \LaTeX{} con las siguientes características:

\begin{itemize}
    \item Diseño profesional de dos columnas
    \item Generación automática de página de título
    \item Gestión avanzada de autores con afiliaciones
    \item Control de versiones y numeración de compilaciones
    \item Formato personalizado de secciones
    \item Cajas de resaltado para contenido importante
    \item Notas al pie en el lado derecho
    \item Todas las características del paquete \inlinecode{pt-commons}
\end{itemize}

\subsection{Opciones de la Clase}

El documento puede configurarse con diferentes opciones:

\begin{enumerate}
    \item \textbf{Idiomas:} spanish, english, portuguese, french
    \item \textbf{Código:} nominted (desactiva minted)
    \item \textbf{Estándar:} twocolumn, onecolumn, 10pt, 11pt, 12pt
\end{enumerate}

\section{Formato de Texto}

\subsection{Estilos Básicos}

El texto puede ser \textbf{negrita}, \textit{cursiva}, \underline{subrayado},
o \texttt{monoespaciado}. También se puede combinar \textbf{\textit{negrita
con cursiva}}.

Para código en línea, use el comando \inlinecode{\textbackslash{}inlinecode\{texto\}}
que produce: \inlinecode{variable\_name}.

\subsection{Párrafos y Espaciado}

La clase configura automáticamente:
\begin{itemize}
    \item Indentación de párrafo: 21pt
    \item Separación de columnas: 21pt
    \item Márgenes: 42pt en todos los lados
\end{itemize}

\subsubsection{Sub-subsecciones}

Las sub-subsecciones tienen formato normal (no negrita) para diferenciarlas
visualmente de las subsecciones.

\paragraph{Títulos de Párrafo}

Los párrafos pueden tener títulos subrayados usando el comando
\inlinecode{\textbackslash{}paragraph}.

\section{Elementos Especiales}

\subsection{Caja de Resaltado}

Use el entorno \inlinecode{highlightbox} para resaltar información importante:

\begin{highlightbox}
\textbf{Nota Importante:} Esta es una caja de resaltado que llama la atención
sobre información clave. Es útil para advertencias, notas importantes, o
conceptos destacados.
\end{highlightbox}

Las cajas de resaltado tienen:
\begin{itemize}
    \item Fondo azul claro (\inlinecode{ptblue!14})
    \item Borde azul (\inlinecode{ptblue})
    \item Esquinas redondeadas
    \item Ancho completo de columna
\end{itemize}

\subsection{Citas}

Para citas largas, use el entorno \inlinecode{quotation}:

\begin{quotation}
``La tipografía es el arte de dotar al lenguaje humano de una forma visual
duradera, y por lo tanto, de una existencia independiente.''

--- Robert Bringhurst, \textit{The Elements of Typographic Style}
\end{quotation}

\section{Listas}

\subsection{Listas con Viñetas}

Las listas con viñetas tienen espaciado optimizado:

\begin{itemize}
    \item Primer elemento de nivel uno
    \item Segundo elemento de nivel uno
    \begin{itemize}
        \item Elemento de nivel dos
        \item Otro elemento de nivel dos
    \end{itemize}
    \item Tercer elemento de nivel uno
\end{itemize}

\subsection{Listas Numeradas}

Las listas numeradas también tienen formato consistente:

\begin{enumerate}
    \item Primer paso del procedimiento
    \item Segundo paso del procedimiento
    \begin{enumerate}
        \item Sub-paso 2.1
        \item Sub-paso 2.2
    \end{enumerate}
    \item Tercer paso del procedimiento
\end{enumerate}

\section{Tablas}

\subsection{Tablas Básicas}

La clase utiliza \inlinecode{tabularray} para tablas mejoradas:

\begin{table}[h]
\centering
\begin{tblr}{colspec={lcc}}
    \tableheader
    Método & Precisión & Tiempo \\
    \hline
    Algoritmo A & 95\% & 10s \\
    Algoritmo B & 97\% & 15s \\
    Algoritmo C & 93\% & 8s \\
\end{tblr}
\caption{Comparación de algoritmos}
\label{tab:comparacion}
\end{table}

\subsection{Tablas Avanzadas}

Tabla con sub-encabezados y celdas personalizadas:

\begin{table}[h]
\centering
\begin{tblr}{
    colspec={lccc},
    width=\columnwidth
}
    \tableheader
    \tablecellcenter Característica & Clase A & Clase B & Clase C \\
    \hline
    \tablesubheader
    \tablecellbold Rendimiento & Alto & Medio & Bajo \\
    Velocidad & 100 & 75 & 50 \\
    Memoria & 512 MB & 256 MB & 128 MB \\
    \hline
    \tablecellbold Total & \tablecellbold 100\% & 75\% & 50\% \\
\end{tblr}
\caption{Características de las clases}
\label{tab:caracteristicas}
\end{table}

\section{Figuras}

\subsection{Figura Simple}

Use el comando \inlinecode{\textbackslash{}ptfigure} para insertar figuras:

\ptfigure{h}{width=0.8\columnwidth}{images/jarjar.jpg}{Figura del personaje Jar Jar Binks}{fig:jarjar}

Las figuras se centran automáticamente y tienen el formato adecuado.

\subsection{Múltiples Figuras}

También puede usar el entorno estándar de figura:

\begin{figure}[h]
    \centering
    \includegraphics[width=0.6\columnwidth]{images/jarjar.jpg}
    \caption{Figura insertada manualmente}
    \label{fig:manual}
\end{figure}

\section{Código Fuente}

\subsection{Código en Línea}

Use \inlinecode{\textbackslash{}inlinecode\{\}} para código en línea:
\inlinecode{def function():} o \inlinecode{int main()}.

\subsection{Bloques de Código}

Con minted habilitado (compile con \inlinecode{-shell-escape}):

\begin{minted}{python}
def fibonacci(n):
    """Calculate Fibonacci number."""
    if n <= 1:
        return n
    return fibonacci(n-1) + fibonacci(n-2)

# Example usage
result = fibonacci(10)
print(f"Fibonacci(10) = {result}")
\end{minted}

\subsection{Código para Impresión}

Use \inlinecode{ptprintcode} para código en blanco y negro con marco:

\begin{ptprintcode}{c}
#include <stdio.h>

int main() {
    printf("Hello, World!\n");
    return 0;
}
\end{ptprintcode}

\section{Matemáticas}

\subsection{Ecuaciones en Línea}

Las ecuaciones en línea se escriben entre \$: La ecuación $E = mc^2$ es famosa.

\subsection{Ecuaciones Numeradas}

\begin{equation}
    \int_{0}^{\infty} e^{-x^2} dx = \frac{\sqrt{\pi}}{2}
    \label{eq:gaussian}
\end{equation}

\subsection{Múltiples Ecuaciones}

\begin{align}
    f(x) &= x^2 + 2x + 1 \\
    f'(x) &= 2x + 2 \\
    f''(x) &= 2
\end{align}

\section{Árboles de Archivos}

La clase incluye soporte para visualización de estructura de directorios:

\begin{filetree}
\begin{ptdirtree}
    \dirtree{%
        .1 \treeiconfirst{proyecto/}.
        .2 \treeicon{src/}.
        .3 \treeicon{main.py}.
        .3 \treeicon{utils.py}.
        .3 \treeicon{config.json}.
        .2 \treeicon{docs/}.
        .3 \treeicon{README.md}.
        .3 \treeicon{tutorial.pdf}.
        .2 \treeicon{tests/}.
        .3 \treeicon{test\_main.py}.
        .2 \treeicon{requirements.txt}.
        .2 \treeicon{.gitignore}.
    }%
\end{ptdirtree}
\caption{Estructura del proyecto}
\label{fig:tree}
\end{filetree}

\section{Colores}

\subsection{Paleta de Colores PT}

La clase proporciona una paleta de colores predefinida:

\begin{itemize}
    \item \textcolor{ptred}{Rojo PT (ptred)}
    \item \textcolor{ptdarkred}{Rojo Oscuro PT (ptdarkred)}
    \item \textcolor{ptlightblue}{Azul Claro PT (ptlightblue)}
    \item \textcolor{ptblue}{Azul PT (ptblue)}
    \item \textcolor{ptdarkblue}{Azul Oscuro PT (ptdarkblue)}
    \item \textcolor{ptgreen}{Verde PT (ptgreen)}
    \item \textcolor{ptdarkgreen}{Verde Oscuro PT (ptdarkgreen)}
    \item \textcolor{ptyellow}{Amarillo PT (ptyellow)}
    \item \textcolor{ptdarkyellow}{Amarillo Oscuro PT (ptdarkyellow)}
    \item \textcolor{ptgray}{Gris PT (ptgray)}
    \item \textcolor{ptdarkgray}{Gris Oscuro PT (ptdarkgray)}
\end{itemize}

\section{Referencias}

\subsection{Referencias Cruzadas}

Puede hacer referencia a secciones, figuras, tablas y ecuaciones:

\begin{itemize}
    \item Ver Figura~\ref{fig:jarjar} para el logo
    \item La Tabla~\ref{tab:comparacion} muestra los resultados
    \item La Ecuación~\ref{eq:gaussian} es la integral gaussiana
    \item Ver Figura~\ref{fig:tree} para la estructura del proyecto
\end{itemize}

\subsection{Notas al Pie}

Las notas al pie\footnote{Esta es una nota al pie que aparece en el lado
derecho de la página.} se colocan automáticamente en el lado derecho.

\section{Balance de Columnas}

Por defecto, las columnas NO se balancean al final del documento (más seguro
para figuras grandes). Para habilitar el balance de columnas:

% \ptbalancecolumns  % Descomente para balancear columnas

Para deshabilitarlo explícitamente:

% \ptnobalancecolumns

\section{Control de Versiones}

\subsection{Numeración de Versiones}

El pie de página muestra automáticamente:
\begin{itemize}
    \item Versión: \docversion
    \item Número de compilación (si está configurado)
    \item Fecha: \todayymd
\end{itemize}

\subsection{Marcas de Agua}

Descomente la línea \inlinecode{\textbackslash{}watermark\{BORRADOR\}} en el
preámbulo para agregar una marca de agua. La marca de agua incluye
automáticamente la versión y el número de compilación.

\section{Compilación}

\subsection{Compilación Estándar}

\begin{minted}{bash}
pdflatex template.tex
\end{minted}

\subsection{Con Minted (Resaltado de Código)}

\begin{minted}{bash}
pdflatex -shell-escape template.tex
\end{minted}

\subsection{Compilación Completa}

Para documentos con bibliografía:

\begin{minted}{bash}
pdflatex -shell-escape template.tex
bibtex template
pdflatex -shell-escape template.tex
pdflatex -shell-escape template.tex
\end{minted}

\section{Consejos y Mejores Prácticas}

\begin{highlightbox}
\textbf{Consejos para Usuarios:}
\begin{itemize}
    \item Use \inlinecode{\textbackslash{}addauthor} para gestión correcta de autores
    \item Habilite balance de columnas solo en versión final
    \item Use cajas de resaltado con moderación
    \item Configure versión y compilación para control de versiones
    \item Use \inlinecode{ptfigure} para consistencia
    \item Compile con \inlinecode{-shell-escape} para minted
    \item Use \inlinecode{\textbackslash{}inlinecode} para fragmentos de código
    \item Deje que la clase maneje el espaciado
\end{itemize}
\end{highlightbox}

\section{Conclusión}

Esta plantilla demuestra todas las capacidades de la clase \inlinecode{pt-article}.
Para más información, consulte:

\begin{itemize}
    \item README.md en el directorio de la clase
    \item Documentación de pt-commons
    \item Repositorio: \url{https://github.com/ptoledo-teaching/pt-commons}
\end{itemize}

% Bibliography (optional)
% \bibliographystyle{plain}
% \bibliography{references}

\end{document}
